% 简体中文短文模板(建议使用 XeLaTeX 编译)
% 若目标会议提供官方 LaTeX 模板,请以官方模板为准再替换正文内容
\documentclass[conference]{IEEEtran}
\usepackage{xeCJK}
\usepackage{times}
\usepackage{graphicx}
\usepackage{amsmath,amssymb}
\usepackage{booktabs}
\usepackage[hidelinks]{hyperref}
\usepackage{siunitx}

\setCJKmainfont{Noto Sans CJK SC} % 可替换为系统可用中文字体
\sisetup{detect-all}

\begin{document}
\title{面向室内目标检测的高效 MobileNetV4+SEA 注意力增强方法}

\author{\IEEEauthorblockN{作者甲, 作者乙}
\IEEEauthorblockA{单位名称 \\
Email: \{first,last\}@example.com}
}

\maketitle

\begin{abstract}
我们在 RT-DETR 框架中提出一种轻量注意力插入方案,将 SEA 模块嵌入 MobileNetV4 主干以提升室内场景目标检测的精度–效率折中。在 HomeObjects-3K 数据集上,所提模型(MNV4-SEA)相较于 MNV4 基线的 mAP50 提升约 \textbf{+14.4\%}(以最终复现实验为准),参数量约 \SI{29.06}{M};在 COCO-Indoor 子集上的验证显示良好跨数据集泛化。我们提供完整的可复现资产(代码、配置与子集生成脚本),并讨论了未来融合 CLIP 的指令驱动目标检测方向。
\end{abstract}

\section{引言}
% 背景与挑战、移动端约束、现有方法不足
\textbf{本文贡献:}
(i) 提出适用于移动友好主干的 SEA 注意力插入范式;
(ii) 系统消融以区分结构增益与训练策略增益;
(iii) 完整效率评测与 COCO-Indoor 跨数据集验证;
(iv) 提供可复现代码与脚本。

\section{相关工作}
% 高效主干、轻量注意力、实时检测、开放词汇(简述)

\section{方法}
% RT-DETR 与 MobileNetV4 概述
% SEA 模块结构、插入位置(stage/block)、超参、复杂度分析
% 图1:结构示意
% 说明为何选择注意力而非融合颈

\section{实验}
\subsection{数据集与指标}
% HomeObjects-3K,COCO-Indoor(Places365筛选脚本);COCO指标(mAP50/95,APs/m/l 等)

\subsection{实现细节}
% 硬件、软件版本、imgsz/batch、LR/调度、EMA、epoch、seed;3 seed 均值±std

\subsection{主结果}
% 表1:HomeObjects-3K 主结果(mAP50/95、参数、FLOPs、延迟)
% 简述相对提升与对比方法

\subsection{跨数据集与消融}
% 表2:COCO-Indoor 验证结果(泛化)
% 表3:插入位置/通道压缩比/EMA/增强/损失等消融
% 表4:分辨率/主干规模(T/S/L)趋势
% 表5:效率(GPU/CPU/端侧延迟,显存峰值);Pareto 曲线

\subsection{可视化与误差分析}
% PR 曲线、混淆矩阵、小/中/大目标 AP 分解
% 预测可视化与失败案例

\section{讨论}
% 何时/为何有效、与融合颈对比、局限与风险

\section{结论与展望}
% 小结;展望:CLIP 指令驱动,OVD 协议(COCO/LVIS base–novel)

\bibliographystyle{IEEEtran}
\bibliography{refs}
\end{document}