% Minimal IEEE-like short paper template (customize to target venue)
\documentclass[conference]{IEEEtran}
\usepackage{times}
\usepackage{graphicx}
\usepackage{amsmath,amssymb}
\usepackage{booktabs}
\usepackage[hidelinks]{hyperref}

\begin{document}
\title{Efficient Attention-Augmented MobileNetV4 for Real-Time Indoor Object Detection}

\author{\IEEEauthorblockN{First Author, Second Author}
\IEEEauthorblockA{Affiliation \\
Email: \{first,last\}@example.com}
}

\maketitle

\begin{abstract}
We propose a lightweight attention insertion for MobileNetV4 within RT-DETR to improve indoor object detection under compute constraints. On HomeObjects-3K, our model (MNV4-SEA) improves mAP50 by \textbf{+14.4\%} over MNV4 baseline with only 29.06M parameters, and shows favorable accuracy–efficiency trade-off compared to popular real-time baselines. Cross-dataset validation on a COCO-Indoor subset confirms generalization. Code, configs, and dataset-generation scripts are released. Future work explores CLIP-instruction-driven detection.
\end{abstract}

\section{Introduction}
% indoor challenges, mobile constraints, gap, contributions
\textbf{Contributions:} (i) mobile-friendly attention insertion scheme; (ii) extensive ablations; (iii) efficiency study and cross-dataset validation; (iv) reproducibility.

\section{Related Work}
% Efficient CNNs (MobileNetV3/V4), Attention (SE/ECA/CBAM/SEA), Real-time detectors (YOLOv8/11, RT-DETR), OVD (Detic, OWL-ViT)

\section{Method}
% RT-DETR + MobileNetV4 recap; SEA module; insertion points; complexity
% Figure 1: architecture diagram

\section{Experiments}
\subsection{Datasets and Metrics}
% HomeObjects-3K; COCO-Indoor subset (script); COCO metrics
\subsection{Implementation Details}
% training schedule, seeds, hardware, versions
\subsection{Main Results}
% Table 1: main benchmark vs baselines (mAP50/95, Params, FLOPs, Latency)
\subsection{Ablations and Efficiency}
% Tables 3-5; PR curves, confusion matrix, qualitative

\section{Discussion}
% when/why SEA helps; limitations

\section{Conclusion and Future Work}
% summary + CLIP instruction-driven plan

\bibliographystyle{IEEEtran}
\bibliography{refs}
\end{document}